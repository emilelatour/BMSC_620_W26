% Options for packages loaded elsewhere
\PassOptionsToPackage{unicode}{hyperref}
\PassOptionsToPackage{hyphens}{url}
\PassOptionsToPackage{dvipsnames,svgnames,x11names}{xcolor}
%
\documentclass[
  letterpaper,
  DIV=11,
  numbers=noendperiod]{scrartcl}

\usepackage{amsmath,amssymb}
\usepackage{iftex}
\ifPDFTeX
  \usepackage[T1]{fontenc}
  \usepackage[utf8]{inputenc}
  \usepackage{textcomp} % provide euro and other symbols
\else % if luatex or xetex
  \usepackage{unicode-math}
  \defaultfontfeatures{Scale=MatchLowercase}
  \defaultfontfeatures[\rmfamily]{Ligatures=TeX,Scale=1}
\fi
\usepackage{lmodern}
\ifPDFTeX\else  
    % xetex/luatex font selection
\fi
% Use upquote if available, for straight quotes in verbatim environments
\IfFileExists{upquote.sty}{\usepackage{upquote}}{}
\IfFileExists{microtype.sty}{% use microtype if available
  \usepackage[]{microtype}
  \UseMicrotypeSet[protrusion]{basicmath} % disable protrusion for tt fonts
}{}
\makeatletter
\@ifundefined{KOMAClassName}{% if non-KOMA class
  \IfFileExists{parskip.sty}{%
    \usepackage{parskip}
  }{% else
    \setlength{\parindent}{0pt}
    \setlength{\parskip}{6pt plus 2pt minus 1pt}}
}{% if KOMA class
  \KOMAoptions{parskip=half}}
\makeatother
\usepackage{xcolor}
\usepackage[margin=0.75in]{geometry}
\setlength{\emergencystretch}{3em} % prevent overfull lines
\setcounter{secnumdepth}{-\maxdimen} % remove section numbering
% Make \paragraph and \subparagraph free-standing
\makeatletter
\ifx\paragraph\undefined\else
  \let\oldparagraph\paragraph
  \renewcommand{\paragraph}{
    \@ifstar
      \xxxParagraphStar
      \xxxParagraphNoStar
  }
  \newcommand{\xxxParagraphStar}[1]{\oldparagraph*{#1}\mbox{}}
  \newcommand{\xxxParagraphNoStar}[1]{\oldparagraph{#1}\mbox{}}
\fi
\ifx\subparagraph\undefined\else
  \let\oldsubparagraph\subparagraph
  \renewcommand{\subparagraph}{
    \@ifstar
      \xxxSubParagraphStar
      \xxxSubParagraphNoStar
  }
  \newcommand{\xxxSubParagraphStar}[1]{\oldsubparagraph*{#1}\mbox{}}
  \newcommand{\xxxSubParagraphNoStar}[1]{\oldsubparagraph{#1}\mbox{}}
\fi
\makeatother

\usepackage{color}
\usepackage{fancyvrb}
\newcommand{\VerbBar}{|}
\newcommand{\VERB}{\Verb[commandchars=\\\{\}]}
\DefineVerbatimEnvironment{Highlighting}{Verbatim}{commandchars=\\\{\}}
% Add ',fontsize=\small' for more characters per line
\usepackage{framed}
\definecolor{shadecolor}{RGB}{241,243,245}
\newenvironment{Shaded}{\begin{snugshade}}{\end{snugshade}}
\newcommand{\AlertTok}[1]{\textcolor[rgb]{0.68,0.00,0.00}{#1}}
\newcommand{\AnnotationTok}[1]{\textcolor[rgb]{0.37,0.37,0.37}{#1}}
\newcommand{\AttributeTok}[1]{\textcolor[rgb]{0.40,0.45,0.13}{#1}}
\newcommand{\BaseNTok}[1]{\textcolor[rgb]{0.68,0.00,0.00}{#1}}
\newcommand{\BuiltInTok}[1]{\textcolor[rgb]{0.00,0.23,0.31}{#1}}
\newcommand{\CharTok}[1]{\textcolor[rgb]{0.13,0.47,0.30}{#1}}
\newcommand{\CommentTok}[1]{\textcolor[rgb]{0.37,0.37,0.37}{#1}}
\newcommand{\CommentVarTok}[1]{\textcolor[rgb]{0.37,0.37,0.37}{\textit{#1}}}
\newcommand{\ConstantTok}[1]{\textcolor[rgb]{0.56,0.35,0.01}{#1}}
\newcommand{\ControlFlowTok}[1]{\textcolor[rgb]{0.00,0.23,0.31}{\textbf{#1}}}
\newcommand{\DataTypeTok}[1]{\textcolor[rgb]{0.68,0.00,0.00}{#1}}
\newcommand{\DecValTok}[1]{\textcolor[rgb]{0.68,0.00,0.00}{#1}}
\newcommand{\DocumentationTok}[1]{\textcolor[rgb]{0.37,0.37,0.37}{\textit{#1}}}
\newcommand{\ErrorTok}[1]{\textcolor[rgb]{0.68,0.00,0.00}{#1}}
\newcommand{\ExtensionTok}[1]{\textcolor[rgb]{0.00,0.23,0.31}{#1}}
\newcommand{\FloatTok}[1]{\textcolor[rgb]{0.68,0.00,0.00}{#1}}
\newcommand{\FunctionTok}[1]{\textcolor[rgb]{0.28,0.35,0.67}{#1}}
\newcommand{\ImportTok}[1]{\textcolor[rgb]{0.00,0.46,0.62}{#1}}
\newcommand{\InformationTok}[1]{\textcolor[rgb]{0.37,0.37,0.37}{#1}}
\newcommand{\KeywordTok}[1]{\textcolor[rgb]{0.00,0.23,0.31}{\textbf{#1}}}
\newcommand{\NormalTok}[1]{\textcolor[rgb]{0.00,0.23,0.31}{#1}}
\newcommand{\OperatorTok}[1]{\textcolor[rgb]{0.37,0.37,0.37}{#1}}
\newcommand{\OtherTok}[1]{\textcolor[rgb]{0.00,0.23,0.31}{#1}}
\newcommand{\PreprocessorTok}[1]{\textcolor[rgb]{0.68,0.00,0.00}{#1}}
\newcommand{\RegionMarkerTok}[1]{\textcolor[rgb]{0.00,0.23,0.31}{#1}}
\newcommand{\SpecialCharTok}[1]{\textcolor[rgb]{0.37,0.37,0.37}{#1}}
\newcommand{\SpecialStringTok}[1]{\textcolor[rgb]{0.13,0.47,0.30}{#1}}
\newcommand{\StringTok}[1]{\textcolor[rgb]{0.13,0.47,0.30}{#1}}
\newcommand{\VariableTok}[1]{\textcolor[rgb]{0.07,0.07,0.07}{#1}}
\newcommand{\VerbatimStringTok}[1]{\textcolor[rgb]{0.13,0.47,0.30}{#1}}
\newcommand{\WarningTok}[1]{\textcolor[rgb]{0.37,0.37,0.37}{\textit{#1}}}

\providecommand{\tightlist}{%
  \setlength{\itemsep}{0pt}\setlength{\parskip}{0pt}}\usepackage{longtable,booktabs,array}
\usepackage{calc} % for calculating minipage widths
% Correct order of tables after \paragraph or \subparagraph
\usepackage{etoolbox}
\makeatletter
\patchcmd\longtable{\par}{\if@noskipsec\mbox{}\fi\par}{}{}
\makeatother
% Allow footnotes in longtable head/foot
\IfFileExists{footnotehyper.sty}{\usepackage{footnotehyper}}{\usepackage{footnote}}
\makesavenoteenv{longtable}
\usepackage{graphicx}
\makeatletter
\def\maxwidth{\ifdim\Gin@nat@width>\linewidth\linewidth\else\Gin@nat@width\fi}
\def\maxheight{\ifdim\Gin@nat@height>\textheight\textheight\else\Gin@nat@height\fi}
\makeatother
% Scale images if necessary, so that they will not overflow the page
% margins by default, and it is still possible to overwrite the defaults
% using explicit options in \includegraphics[width, height, ...]{}
\setkeys{Gin}{width=\maxwidth,height=\maxheight,keepaspectratio}
% Set default figure placement to htbp
\makeatletter
\def\fps@figure{htbp}
\makeatother

\KOMAoption{captions}{tableheading}
\makeatletter
\@ifpackageloaded{caption}{}{\usepackage{caption}}
\AtBeginDocument{%
\ifdefined\contentsname
  \renewcommand*\contentsname{Table of contents}
\else
  \newcommand\contentsname{Table of contents}
\fi
\ifdefined\listfigurename
  \renewcommand*\listfigurename{List of Figures}
\else
  \newcommand\listfigurename{List of Figures}
\fi
\ifdefined\listtablename
  \renewcommand*\listtablename{List of Tables}
\else
  \newcommand\listtablename{List of Tables}
\fi
\ifdefined\figurename
  \renewcommand*\figurename{Figure}
\else
  \newcommand\figurename{Figure}
\fi
\ifdefined\tablename
  \renewcommand*\tablename{Table}
\else
  \newcommand\tablename{Table}
\fi
}
\@ifpackageloaded{float}{}{\usepackage{float}}
\floatstyle{ruled}
\@ifundefined{c@chapter}{\newfloat{codelisting}{h}{lop}}{\newfloat{codelisting}{h}{lop}[chapter]}
\floatname{codelisting}{Listing}
\newcommand*\listoflistings{\listof{codelisting}{List of Listings}}
\makeatother
\makeatletter
\makeatother
\makeatletter
\@ifpackageloaded{caption}{}{\usepackage{caption}}
\@ifpackageloaded{subcaption}{}{\usepackage{subcaption}}
\makeatother

\ifLuaTeX
  \usepackage{selnolig}  % disable illegal ligatures
\fi
\usepackage{bookmark}

\IfFileExists{xurl.sty}{\usepackage{xurl}}{} % add URL line breaks if available
\urlstyle{same} % disable monospaced font for URLs
\hypersetup{
  pdftitle={Hypothesis Testing: One-Page Reference},
  colorlinks=true,
  linkcolor={blue},
  filecolor={Maroon},
  citecolor={Blue},
  urlcolor={Blue},
  pdfcreator={LaTeX via pandoc}}


\title{Hypothesis Testing: One-Page Reference}
\usepackage{etoolbox}
\makeatletter
\providecommand{\subtitle}[1]{% add subtitle to \maketitle
  \apptocmd{\@title}{\par {\large #1 \par}}{}{}
}
\makeatother
\subtitle{BMSC 620}
\author{}
\date{}

\begin{document}
\maketitle


\subsection{What is hypothesis
testing?}\label{what-is-hypothesis-testing}

\textbf{Goal:} Use sample data to evaluate evidence against a specific
claim about a population parameter.

\begin{itemize}
\tightlist
\item
  The claim we test is the \textbf{null hypothesis}
\item
  We assume the null is true and ask: \emph{How surprising is our data
  under that assumption?}
\end{itemize}

\begin{center}\rule{0.5\linewidth}{0.5pt}\end{center}

\subsection{Key hypotheses}\label{key-hypotheses}

\textbf{Null hypothesis (\(H_0\))}

\begin{itemize}
\tightlist
\item
  Represents the status quo or specific claim
\item
  Uses an equals sign
\end{itemize}

\[H_0: \mu = \mu_0\]

\textbf{Alternative hypothesis (\(H_A\))}

\begin{itemize}
\tightlist
\item
  Represents what we're looking for evidence in favor of
\item
  Uses \(\neq\), \(<\), or \(>\)
\end{itemize}

\[H_A: \mu \neq \mu_0 \quad \text{(two-sided)}\]
\[H_A: \mu < \mu_0 \quad \text{or} \quad H_A: \mu > \mu_0 \quad \text{(one-sided)}\]

\begin{center}\rule{0.5\linewidth}{0.5pt}\end{center}

\subsection{\texorpdfstring{Significance level
(\(\alpha\))}{Significance level (\textbackslash alpha)}}\label{significance-level-alpha}

\begin{itemize}
\tightlist
\item
  \(\alpha\) is the threshold for ``strong evidence''
\item
  Chosen \textbf{before} seeing the data
\item
  Most common: \(\alpha = 0.05\)
\end{itemize}

\textbf{Interpretation:} If \(H_0\) is true, we are willing to reject it
incorrectly at most \(\alpha \times 100\)\% of the time.

\begin{center}\rule{0.5\linewidth}{0.5pt}\end{center}

\subsection{Assumptions for a one-sample
t-test}\label{assumptions-for-a-one-sample-t-test}

\begin{enumerate}
\def\labelenumi{\arabic{enumi}.}
\tightlist
\item
  Observations are \textbf{independent}
\item
  Data are approximately \textbf{normal} OR sample size is large
  (\(n \geq 30\), CLT applies)
\end{enumerate}

\begin{center}\rule{0.5\linewidth}{0.5pt}\end{center}

\subsection{Test statistic (what comes from the
data)}\label{test-statistic-what-comes-from-the-data}

The t-statistic measures how far the sample mean is from the null value,
in standard error units:

\[t = \frac{\bar{x} - \mu_0}{s / \sqrt{n}}\]

\begin{itemize}
\tightlist
\item
  Comes from the \textbf{sample}
\item
  \textbf{Random} (varies from study to study)
\item
  Under \(H_0\), follows a t-distribution with \(df = n - 1\)
\end{itemize}

\begin{center}\rule{0.5\linewidth}{0.5pt}\end{center}

\subsection{Critical value (what defines ``too
extreme'')}\label{critical-value-what-defines-too-extreme}

\begin{itemize}
\tightlist
\item
  Comes from the \textbf{t-distribution}
\item
  Depends on \(\alpha\) and degrees of freedom
\item
  \textbf{Fixed} before seeing the data
\end{itemize}

\[t^* = t_{1-\alpha/2,\; df} \quad \text{(two-sided)}\]

\begin{center}\rule{0.5\linewidth}{0.5pt}\end{center}

\subsection{Decision rules (three equivalent
ways)}\label{decision-rules-three-equivalent-ways}

\subsubsection{1. Test-statistic
approach}\label{test-statistic-approach}

\textbf{Reject \(H_0\) if:} \(|t_{\text{obs}}| > t^*\)

\subsubsection{2. P-value approach}\label{p-value-approach}

\textbf{P-value:} Probability of observing a test statistic as extreme
as (or more extreme than) what we saw, assuming \(H_0\) is true.

\textbf{Reject \(H_0\) if:} \(\text{p-value} < \alpha\)

\subsubsection{3. Confidence interval approach (two-sided
tests)}\label{confidence-interval-approach-two-sided-tests}

\textbf{Reject \(H_0\) if:} the \((1 - \alpha) \times 100\)\% confidence
interval does \textbf{not} contain \(\mu_0\)

\begin{center}\rule{0.5\linewidth}{0.5pt}\end{center}

\subsection{What p-values mean (and don't
mean)}\label{what-p-values-mean-and-dont-mean}

\textbf{P-value IS:}

\begin{itemize}
\tightlist
\item
  A measure of how surprising the data are if \(H_0\) were true
\end{itemize}

\textbf{P-value is NOT:}

\begin{itemize}
\tightlist
\item
  The probability that \(H_0\) is true
\item
  The probability you made a mistake
\item
  A measure of effect size or importance
\end{itemize}

\begin{center}\rule{0.5\linewidth}{0.5pt}\end{center}

\subsection{Common language to use}\label{common-language-to-use}

✔️ ``We reject the null hypothesis''\\
✔️ ``We fail to reject the null hypothesis''

❌ ``We accept the null hypothesis''

\begin{center}\rule{0.5\linewidth}{0.5pt}\end{center}

\subsection{One-sample t-test in R}\label{one-sample-t-test-in-r}

\begin{Shaded}
\begin{Highlighting}[]
\FunctionTok{t.test}\NormalTok{(x, }\AttributeTok{mu =}\NormalTok{ mu0, }\AttributeTok{alternative =} \StringTok{"two.sided"}\NormalTok{, }\AttributeTok{conf.level =} \FloatTok{0.95}\NormalTok{)}
\end{Highlighting}
\end{Shaded}

\textbf{Key output:}

\begin{itemize}
\tightlist
\item
  t-statistic
\item
  degrees of freedom
\item
  p-value
\item
  confidence interval
\item
  sample mean
\end{itemize}

\begin{center}\rule{0.5\linewidth}{0.5pt}\end{center}

\subsection{Reporting results
(example)}\label{reporting-results-example}

\begin{quote}
A one-sample t-test was conducted to assess whether mean body
temperature differs from 98.6°F. The sample (\(n = 130\)) had a mean of
98.25°F (SD = 0.733). The test was statistically significant,
\(t(129) = -5.45\), \(p < 0.001\), with a 95\% confidence interval of
{[}98.12, 98.38{]}, indicating that the population mean body temperature
is lower than 98.6°F.
\end{quote}

\begin{center}\rule{0.5\linewidth}{0.5pt}\end{center}

\subsection{Big picture reminder}\label{big-picture-reminder}

\begin{itemize}
\tightlist
\item
  Hypothesis tests and confidence intervals use the \textbf{same
  information}
\item
  A small p-value does \textbf{not} imply practical importance
\item
  Always \textbf{interpret results in context} ```
\end{itemize}




\end{document}
