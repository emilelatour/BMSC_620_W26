% Options for packages loaded elsewhere
\PassOptionsToPackage{unicode}{hyperref}
\PassOptionsToPackage{hyphens}{url}
\PassOptionsToPackage{dvipsnames,svgnames,x11names}{xcolor}
%
\documentclass[
  letterpaper,
  DIV=11,
  numbers=noendperiod]{scrartcl}

\usepackage{amsmath,amssymb}
\usepackage{iftex}
\ifPDFTeX
  \usepackage[T1]{fontenc}
  \usepackage[utf8]{inputenc}
  \usepackage{textcomp} % provide euro and other symbols
\else % if luatex or xetex
  \usepackage{unicode-math}
  \defaultfontfeatures{Scale=MatchLowercase}
  \defaultfontfeatures[\rmfamily]{Ligatures=TeX,Scale=1}
\fi
\usepackage{lmodern}
\ifPDFTeX\else  
    % xetex/luatex font selection
\fi
% Use upquote if available, for straight quotes in verbatim environments
\IfFileExists{upquote.sty}{\usepackage{upquote}}{}
\IfFileExists{microtype.sty}{% use microtype if available
  \usepackage[]{microtype}
  \UseMicrotypeSet[protrusion]{basicmath} % disable protrusion for tt fonts
}{}
\makeatletter
\@ifundefined{KOMAClassName}{% if non-KOMA class
  \IfFileExists{parskip.sty}{%
    \usepackage{parskip}
  }{% else
    \setlength{\parindent}{0pt}
    \setlength{\parskip}{6pt plus 2pt minus 1pt}}
}{% if KOMA class
  \KOMAoptions{parskip=half}}
\makeatother
\usepackage{xcolor}
\setlength{\emergencystretch}{3em} % prevent overfull lines
\setcounter{secnumdepth}{-\maxdimen} % remove section numbering
% Make \paragraph and \subparagraph free-standing
\makeatletter
\ifx\paragraph\undefined\else
  \let\oldparagraph\paragraph
  \renewcommand{\paragraph}{
    \@ifstar
      \xxxParagraphStar
      \xxxParagraphNoStar
  }
  \newcommand{\xxxParagraphStar}[1]{\oldparagraph*{#1}\mbox{}}
  \newcommand{\xxxParagraphNoStar}[1]{\oldparagraph{#1}\mbox{}}
\fi
\ifx\subparagraph\undefined\else
  \let\oldsubparagraph\subparagraph
  \renewcommand{\subparagraph}{
    \@ifstar
      \xxxSubParagraphStar
      \xxxSubParagraphNoStar
  }
  \newcommand{\xxxSubParagraphStar}[1]{\oldsubparagraph*{#1}\mbox{}}
  \newcommand{\xxxSubParagraphNoStar}[1]{\oldsubparagraph{#1}\mbox{}}
\fi
\makeatother


\providecommand{\tightlist}{%
  \setlength{\itemsep}{0pt}\setlength{\parskip}{0pt}}\usepackage{longtable,booktabs,array}
\usepackage{calc} % for calculating minipage widths
% Correct order of tables after \paragraph or \subparagraph
\usepackage{etoolbox}
\makeatletter
\patchcmd\longtable{\par}{\if@noskipsec\mbox{}\fi\par}{}{}
\makeatother
% Allow footnotes in longtable head/foot
\IfFileExists{footnotehyper.sty}{\usepackage{footnotehyper}}{\usepackage{footnote}}
\makesavenoteenv{longtable}
\usepackage{graphicx}
\makeatletter
\def\maxwidth{\ifdim\Gin@nat@width>\linewidth\linewidth\else\Gin@nat@width\fi}
\def\maxheight{\ifdim\Gin@nat@height>\textheight\textheight\else\Gin@nat@height\fi}
\makeatother
% Scale images if necessary, so that they will not overflow the page
% margins by default, and it is still possible to overwrite the defaults
% using explicit options in \includegraphics[width, height, ...]{}
\setkeys{Gin}{width=\maxwidth,height=\maxheight,keepaspectratio}
% Set default figure placement to htbp
\makeatletter
\def\fps@figure{htbp}
\makeatother

\KOMAoption{captions}{tableheading}
\makeatletter
\@ifpackageloaded{caption}{}{\usepackage{caption}}
\AtBeginDocument{%
\ifdefined\contentsname
  \renewcommand*\contentsname{Table of contents}
\else
  \newcommand\contentsname{Table of contents}
\fi
\ifdefined\listfigurename
  \renewcommand*\listfigurename{List of Figures}
\else
  \newcommand\listfigurename{List of Figures}
\fi
\ifdefined\listtablename
  \renewcommand*\listtablename{List of Tables}
\else
  \newcommand\listtablename{List of Tables}
\fi
\ifdefined\figurename
  \renewcommand*\figurename{Figure}
\else
  \newcommand\figurename{Figure}
\fi
\ifdefined\tablename
  \renewcommand*\tablename{Table}
\else
  \newcommand\tablename{Table}
\fi
}
\@ifpackageloaded{float}{}{\usepackage{float}}
\floatstyle{ruled}
\@ifundefined{c@chapter}{\newfloat{codelisting}{h}{lop}}{\newfloat{codelisting}{h}{lop}[chapter]}
\floatname{codelisting}{Listing}
\newcommand*\listoflistings{\listof{codelisting}{List of Listings}}
\makeatother
\makeatletter
\makeatother
\makeatletter
\@ifpackageloaded{caption}{}{\usepackage{caption}}
\@ifpackageloaded{subcaption}{}{\usepackage{subcaption}}
\makeatother

\ifLuaTeX
  \usepackage{selnolig}  % disable illegal ligatures
\fi
\usepackage{bookmark}

\IfFileExists{xurl.sty}{\usepackage{xurl}}{} % add URL line breaks if available
\urlstyle{same} % disable monospaced font for URLs
\hypersetup{
  pdftitle={Syllabus},
  colorlinks=true,
  linkcolor={blue},
  filecolor={Maroon},
  citecolor={Blue},
  urlcolor={Blue},
  pdfcreator={LaTeX via pandoc}}


\title{Syllabus}
\author{}
\date{}

\begin{document}
\maketitle


\section{Key course info}\label{key-course-info}

\begin{itemize}
\tightlist
\item
  \textbf{Course website}: This site is the hub for schedules and
  materials. Assignments and grades will be managed in \textbf{Sakai}.
\item
  \textbf{Software}: We will use \textbf{R and RStudio} for homework and
  in-class examples. (Instructions will be posted on the Resources
  page.)
\item
  \textbf{One-day room change}: Class will meet in \textbf{MAC 3198 on
  January} 21 only.
\end{itemize}

\section{Course communication}\label{course-communication}

\textbf{Announcements}\\
For important or time-sensitive information (e.g., due date changes,
clarifications, reminders), I will post announcements on \textbf{Sakai}.
These announcements will also be delivered to your OHSU email address.

\textbf{General course questions}\\
It is normal to have questions about assignments, course materials, or
logistics. Before emailing me, please check the course website and
recent announcements, as many questions are shared by multiple students.

If your question is still unanswered, you are welcome to email me.

\textbf{Email}\\
Email is the best way to reach me for course-related questions. I aim to
respond within \textbf{24 hours, Monday--Friday}. I do not regularly
monitor email on evenings or weekends.

For privacy reasons, questions about grades or personal circumstances
should always be sent by email.

\section{Course description}\label{course-description}

BMSC 620 introduces fundamental statistical methods commonly used in the
biomedical and health sciences, with an emphasis on understanding,
interpretation, and clear communication rather than mathematical
derivations.

We will begin with descriptive statistics and graphical methods for
summarizing data, followed by basic probability concepts that motivate
statistical inference. Key probability and sampling distributions,
including the binomial, Poisson, and normal distributions, will be
introduced as tools for understanding variability and uncertainty in
data.

The course will then cover confidence intervals and hypothesis testing
for one- and two-sample problems using parametric and nonparametric
approaches. Additional topics include inference for proportions,
analysis of two-way tables, one-way analysis of variance (ANOVA),
correlation, and simple linear regression.

Throughout the course, emphasis will be placed on selecting appropriate
methods, interpreting results, and communicating conclusions in a way
that is accessible to audiences without formal statistical training.
Students will gain hands-on experience using R for basic data
management, visualization, and interpretation of statistical output.

\section{Learning objectives}\label{learning-objectives}

By the end of this course, students should be able to:

\begin{enumerate}
\def\labelenumi{\arabic{enumi}.}
\tightlist
\item
  Describe how data relate to a research question of interest.
\item
  Select and implement appropriate statistical methods for estimation
  and inference using statistical software.
\item
  Interpret and communicate statistical findings to a non-statistical
  audience in the context of the original research question.
\end{enumerate}

\section{Instructors and office
hours}\label{instructors-and-office-hours}

See the \href{instructors.qmd}{Instructors} page for contact information
and office hours.

\section{Meeting times and location}\label{meeting-times-and-location}

\begin{itemize}
\tightlist
\item
  \textbf{Days}: Mondays and Wednesdays
\item
  \textbf{Dates}: January 5 -- March 20, 2026
\item
  \textbf{Time}: 9:00 AM -- 11:30 AM
\item
  \textbf{Location}: RJH 4320 (except January 21 in MAC 3198)
\end{itemize}

\subsubsection{Known exceptions}\label{known-exceptions}

\begin{itemize}
\tightlist
\item
  Monday, January 19, 2026: Martin Luther King Jr.~Day
\item
  Monday, February 16, 2026: Presidents Day
\end{itemize}

\section{Materials and software}\label{materials-and-software}

See the \href{resources.qmd}{Resources} page for links to software
installation instructions, course handouts, datasets, and other
resources.

\section{Assessment and grading}\label{assessment-and-grading}

Assessment in this course is designed to support learning, practice, and
clear communication of statistical ideas. Grades are based on
\textbf{homework (40\%), a take-home midterm exam (20\%), a take-home
final exam (30\%), and attendance via post-class surveys (10\%)}. The
emphasis throughout is on understanding, interpretation, and reasoning
rather than memorization or speed.

\subsection{Homework (40\%)}\label{homework-40}

Homework assignments are designed to help you practice core concepts,
develop familiarity with R, and build confidence interpreting
statistical results. Homework is primarily a \textbf{formative
assessment}, meaning its purpose is to support learning rather than to
assess perfection.

There will be regular homework assignments throughout the term.
Assignments will typically include a mix of conceptual questions,
interpretation, and applied work using R.

\paragraph{Grading}\label{grading}

Each homework assignment will be graded on a 10-point scale using the
following criteria:

\begin{itemize}
\item
  \textbf{Substantial completion (6 points)}

  Most required parts of the assignment are attempted (approximately
  75--80\%). Code and written responses show a genuine effort to engage
  with the problems, even if some answers are incorrect.
\item
  \textbf{Demonstrated process (2 points)}

  Relevant work is shown, including R code and intermediate steps where
  appropriate. The focus is on showing how you approached the problem
  rather than arriving at a perfect solution.
\item
  \textbf{Interpretation and communication (2 points)}

  Answers include written interpretation where appropriate, with an
  attempt to explain results in context (e.g., direction, magnitude,
  units, or meaning of statistical output).
\end{itemize}

Homework will be graded primarily for \textbf{effort, reasoning, and
clarity}, not strict correctness.

\paragraph{Feedback}\label{feedback}

For homework submitted on time, the TA will provide feedback on one or
more problems. Solutions will be posted after the due date so that you
can review your work and identify areas for improvement.

\paragraph{Late and dropped homework}\label{late-and-dropped-homework}

\begin{itemize}
\tightlist
\item
  Late homework will be accepted for \textbf{partial credit (up to 80\%
  of the total points)}. Feedback will \textbf{only} be provided for
  homework submitted on time.
\item
  The \textbf{lowest homework score will be dropped} when calculating
  the final homework grade. \textbf{HW 0 is excluded from this policy},
  as it is intended to ensure technical readiness for the course.
\end{itemize}

These policies are intended to provide flexibility for busy schedules
while encouraging consistent engagement with the course material.

\subsection{Midterm exam (20\%)}\label{midterm-exam-20}

The midterm exam is a \textbf{take-home assessment} designed to evaluate
your understanding of the material covered in the first half of the
course. The emphasis is on \textbf{conceptual understanding,
interpretation, and clear communication}, rather than speed or
memorization.

The midterm will be available for several days, with the exact release
and due dates listed on the course schedule and in Sakai. The exam is
designed to take approximately \textbf{3--5} hours to complete, though
you may choose when and how to distribute that time during the
availability window.

\paragraph{Format and resources}\label{format-and-resources}

The midterm will consist of a small number of multi-part questions and
may include conceptual questions, interpretation of statistical output
or visualizations, and applied tasks using R.

You may use:

\begin{itemize}
\tightlist
\item
  Course notes and slides
\item
  The course website
\item
  R and RStudio
\end{itemize}

\paragraph{Collaboration and academic
integrity}\label{collaboration-and-academic-integrity}

You may discuss \textbf{general concepts} related to the exam with
classmates. However, you must work \textbf{independently} on the midterm
and submit your own work. Sharing code, written responses, or specific
solutions is not permitted.

\paragraph{Use of AI tools}\label{use-of-ai-tools}

Generative AI tools (e.g., ChatGPT) may be used for limited support such
as clarifying R error messages or syntax. They may not be used to
generate solutions, interpret results, or write responses for the exam.
Any use of AI tools must be consistent with the course's academic
integrity policy.

\paragraph{Submission and grading}\label{submission-and-grading}

You will submit your midterm exam via Sakai. Instructions regarding file
formats and submission details will be provided with the exam.

The midterm will be graded based on correctness, reasoning, and clarity
of interpretation. Partial credit will be awarded where appropriate.

\paragraph{Late policy}\label{late-policy}

\begin{itemize}
\tightlist
\item
  \textbf{Extensions} may be granted \textbf{by request} and should be
  discussed with me as soon as possible.
\item
  \textbf{Unexcused late submissions} will be penalized by \textbf{10\%
  per day late}.
\end{itemize}

\subsection{Final exam (30\%)}\label{final-exam-30}

The final exam is a \textbf{take-home assessment} designed to evaluate
your ability to synthesize course concepts and communicate statistical
results clearly and accurately. The exam is \textbf{cumulative}, with an
emphasis on \textbf{material covered after the midterm}. As with the
midterm, the emphasis is on \textbf{understanding, interpretation, and
reasoning}, rather than speed or memorization.

The final exam will be available for several days, with the exact
release and due dates listed on the course schedule and in Sakai. The
exam is designed to take approximately \textbf{4--6 hours} to complete,
though you may choose when and how to distribute that time during the
availability window.

\paragraph{Format and resources}\label{format-and-resources-1}

The final exam will consist of a small number of multi-part questions
that may include:

\begin{itemize}
\tightlist
\item
  Interpretation of numerical summaries, confidence intervals,
  hypothesis tests, or regression output
\item
  Interpretation of data visualizations
\item
  Applied tasks using R
\item
  Questions that require integrating multiple concepts covered
  throughout the course
\end{itemize}

You may use:

\begin{itemize}
\tightlist
\item
  Course notes and slides
\item
  The course website
\item
  R and RStudio
\end{itemize}

\paragraph{Collaboration and academic
integrity}\label{collaboration-and-academic-integrity-1}

You may discuss \textbf{general concepts} related to the exam with
classmates. However, you must work \textbf{independently} on the final
exam and submit your own work. Sharing code, written responses, or
specific solutions is not permitted.

\paragraph{Use of AI tools}\label{use-of-ai-tools-1}

Generative AI tools (e.g., ChatGPT) may be used for limited support such
as clarifying R error messages or syntax. They may not be used to
generate solutions, interpret results, or write responses for the exam.
Any use of AI tools must be consistent with the course's academic
integrity policy.

\paragraph{Submission and grading}\label{submission-and-grading-1}

You will submit your final exam via Sakai. Instructions regarding file
formats and submission details will be provided with the exam.

The final exam will be graded based on correctness, reasoning, and
clarity of interpretation. Partial credit will be awarded where
appropriate.

\paragraph{Late policy}\label{late-policy-1}

\begin{itemize}
\tightlist
\item
  \textbf{Extensions} may be granted \textbf{by request} and should be
  discussed with me as soon as possible.
\item
  \textbf{Unexcused late submissions} will be penalized by \textbf{10\%
  per day late}.
\end{itemize}

\subsection{Attendance and post-class surveys
(10\%)}\label{attendance-and-post-class-surveys-10}

Attendance and engagement will be assessed through brief
\textbf{post-class surveys} submitted after class meetings. These
surveys are intended to encourage reflection on the day's material and
to provide feedback that helps improve the course.

Surveys may include short questions about key takeaways, points of
confusion, or course logistics. Surveys are designed to be low-effort
and should take only a few minutes to complete.

\paragraph{Credit and expectations}\label{credit-and-expectations}

\begin{itemize}
\tightlist
\item
  You will receive full credit for completing at least \textbf{15 out of
  20} post-class surveys during the term.
\item
  Surveys are graded for \textbf{completion}, not correctness.
\item
  No make-up surveys will be offered, but the 15-of-20 policy is
  intended to provide flexibility for absences, illness, or scheduling
  conflicts.
\end{itemize}

\paragraph{Purpose}\label{purpose}

Post-class surveys serve two purposes:

\begin{enumerate}
\def\labelenumi{\arabic{enumi}.}
\tightlist
\item
  They encourage regular engagement with course material.
\item
  They allow me to identify common questions or areas of confusion and
  adjust instruction accordingly.
\end{enumerate}

This component is designed to reward consistent participation without
penalizing occasional absences.

\section{Collaboration and academic
integrity}\label{collaboration-and-academic-integrity-2}

You are encouraged to discuss course concepts and work through problems
with classmates. However, anything you submit must reflect your own
understanding and be written in your own words.

\subsubsection{Statement on Generative
AI}\label{statement-on-generative-ai}

ChatGPT and other generative AI tools can be great resources for
learning how to code and/or troubleshoot code that does not work.
However, the work you turn in must be your own. Thus it is inappropriate
to rely on AI to directly provide you with solutions to assessment
questions (homework and exams) or write text that you are submitting as
your own. If you do use AI tools to help you with an assignment, these
must be cited along with how they were used.

Please see the Plagiarism \& Attribution section (Code Snippets and AI
Tools subsection) of
\href{https://dmice.ohsu.edu/bedricks/courses/bmi525/policies.html}{Dr.~Steve
Bedrick's Course Policies site} for BMI 525: Principles and Practice of
Data Visualization.

\section{Course policies and university
resources}\label{course-policies-and-university-resources}

\subsubsection{Respect for diversity and
inclusion}\label{respect-for-diversity-and-inclusion}

It is my intent that students from all backgrounds and perspectives be
well served by this course, and that the diversity students bring be
viewed as a strength and resource for learning. Course materials and
activities are designed to be inclusive and respectful of differences in
background, identity, experience, and ways of learning.

I expect all students to engage with one another in a professional,
respectful, and supportive manner. If you have suggestions for how this
course can better support your learning or the learning of others, I
welcome your feedback.

If any class meetings or deadlines conflict with religious, cultural, or
personal obligations, please let me know so we can discuss appropriate
accommodations.

If you have a preferred name or pronouns that differ from those listed
on the course roster, please feel free to share them with me.

\subsubsection{Getting help and asking
questions}\label{getting-help-and-asking-questions}

Questions are a normal and expected part of learning statistics.

When asking for help (in office hours or by email), it is helpful to
include some context about what you have already tried. For example:

\begin{itemize}
\tightlist
\item
  What part of the problem you understand
\item
  Where you are getting stuck
\item
  Any relevant code, output, or notes you are working from
\end{itemize}

Providing context helps me or the TA give more targeted and efficient
feedback. You do not need to have a fully formed question to attend
office hours.

\subsubsection{Copyright of course
materials}\label{copyright-of-course-materials}

All course materials (including slides, handouts, assignments, and
exams) are copyrighted unless otherwise noted. These materials are
intended for use only by students enrolled in this course and may not be
shared or distributed outside the course without permission.

This includes publicly posting course materials or completed assignments
online.

Additional institutional copyright policies are provided elsewhere in
Sakai.

\subsubsection{University resources}\label{university-resources}

Official institutional policies and student support resources (including
accessibility services and academic support) are provided through Sakai.
Please contact me if you have questions or need assistance accessing
these resources.

\section{Schedule}\label{schedule}

See the \href{schedule.qmd}{Schedule} page for the current schedule. The
schedule may be adjusted as needed.

\begin{center}\rule{0.5\linewidth}{0.5pt}\end{center}

\emph{Attribution note}: This syllabus structure was adapted from
departmental course materials, including syllabi by \textbf{Meike
Niederhausen} and \textbf{Nicky Wakim}, with permission, and modified
for BMSC 620.




\end{document}
